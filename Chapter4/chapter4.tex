%%%%%%%%%%%%%%%%%%%%%%%%%chapter4%%%%%%%%%%%%%%%%%%%%%%%%%%%
\chapter{EXPERIMENT AND ANALYSIS}
Uraian pada bab ini meliputi parameter eksperimen, karakteristik data, skenario ujicoba, tempat dan waktu eksperimen, spesifikasi peralatan ujicoba, cara penafsiran dan penyimpulan hasil tesis. Untuk tesis yang menggunakan metode kualitatif, dapat dijelaskan pendekatan yang digunakan, proses pengumpulan dan analisis informasi, proses penafsiran, dan penyimpulan hasil tesis. Analisis hasil eksperimen seharusnya dihubungkan kembali dengan permasalahan. Berikut contoh sistematika penulisan Bab 4.
\section{Experiment Parameter}
Disini penulis dapat membahas parameter eksperimen lebih terperinci. Deskripsi pembahasan seharusnya singkat, padat dan jelas, sehingga membuat pembaca memahami maksud Penulis yang tertuang dalam tulisan.
\section{TEMPAT UJICOBA}
Disini penulis dapat membahas tempat pelaksanaan ujicoba lebih terperinci. Deskripsi pembahasan seharusnya singkat, padat dan jelas, sehingga membuat pembaca memahami maksud penulis yang tertuang dalam tulisan.
\section{WAKTU UJICOBA}
Disini penulis dapat membahas waktu pelaksanaan ujicoba lebih terperinci. Deskripsi pembahasan seharusnya singkat, padat dan jelas, sehingga membuat pembaca memahami maksud penulis yang tertuang dalam tulisan.
\section{SPESIFIKASI PERALATAN UJICOBA}
Disini penulis dapat membahas spesifikasi peralatan lebih terperinci. Peralatan dapat berupa perangkat keras, perangkat lunak, piranti elektronik, gadget, dan lain-lain, yang berfungsi sebagai alat yang dipakai untuk percobaan. Deskripsi pembahasan seharusnya singkat, padat dan jelas, sehingga membuat pembaca memahami maksud penulis yang tertuang dalam tulisan.
\section{HASIL EKSPERIMEN}
Disini penulis dapat membahas hasil eksperimen lebih terperinci. Deskripsi pembahasan seharusnya singkat, padat dan jelas, sehingga membuat pembaca memahami maksud penulis yang tertuang dalam tulisan.
\section{ANALISIS HASIL EKSPERIMEN}
Disini penulis dapat membahas analisis hasil eksperimen lebih terperinci. Analisis hasil eksperimen seharusnya menjawab hipotesa bahwa solusi yang diajukan pada Tujuan dapat menyelesaikan permasalahan. Deskripsi pembahasan seharusnya singkat, padat dan jelas, sehingga membuat pembaca memahami maksud penulis yang tertuang dalam tulisan. 
