%!TEX root = ../thesis.tex
%*******************************************************************************
%*********************************** First Chapter *****************************
%*******************************************************************************

\chapter{INTRODUCTION}  %Title of the First Chapter

\ifpdf
    \graphicspath{{Chapter1/Figs/Raster/}{Chapter1/Figs/PDF/}{Chapter1/Figs/}}
\else
    \graphicspath{{Chapter1/Figs/Vector/}{Chapter1/Figs/}}
\fi


%********************************** %First Section  **************************************
\section{Background} %Section - 1.1 
Deskripsikan latar belakang dari permasalahan yang akan diangkat pada penelitian tesis. Latar belakang berisi penjelasan dari Problem Domain yang termuat pada  judul  penelitian. Misalkan penulis mengambil suatu judul berikut "Deteksi Penyakit Kanker Dengan Sistem Pakar Berbasis Fuzzy". Judul tersebut mempunyai Problem adalah Deteksi Penyakit Kanker, Problem Domain adalah Penyakit Kanker, dan Uniqueness adalah Sistem Pakar Berbasis Fuzzy.
Berkaitan dengan Latar Belakang, Problem Domain-nya adalah tentang penyakit kanker, sehingga disini penulis dapat menceritakan tentang penyakit kanker dan tingkat urgensi (seperti tingkat kenaikan jumlah penderitanya dari tahun ke tahun, gawatnya penyakit kanker tersebut, dan semisalnya). Latar belakang yang baik berisikan problem domain yang mempunyai tingkat urgensi tinggi.


% Please add the following required packages to your document preamble:
% \usepackage{graphicx}
\begin{table}[h!]
\centering
\caption{Comparison Between Virtual Reality With Augmented Reality.}
\label{table:vrar}
\resizebox{15cm}{!}{%
\begin{tabular}{|p{0.2\linewidth}|p{0.45\linewidth}|p{0.45\linewidth}|}\hline
Difference &
  AR &
  VR \\\hline
Environment &
  Combining   the real and virtual worlds. Thus, objects that coexist in the same space and   real-time are a reality. &
  VR environment requires an immersive virtual world environment that   replaces the real world. \\\hline
User Views &
  Allows   the user to see the real world around him and virtual objects &
  Users only see the virtual environment \\\hline
Health Point of View &
  AR solves the problem of "motion   sickness" through superimposing virtual images in a real environment   through special markers where the human brain is still able to process and   accept ideas &
  Known as "motion sickness- where the   human brain cannot distinguish between virtual and real and causes nausea and   severe headaches when adapting \\\hline
Security &
  The user feels comfortable and able to   control the environment. &
  Users feel uncomfortable because their view   is blocked by the virtual environment \\\hline
Immersion Sensation &
  None - Low &
  Medium   - High\\\hline
\end{tabular}%
}
\end{table}


Augmented Reality was chosen with the following considerations, which can be seen in the Table \ref{table:vrar}. This module is embedded in an application with a choice of operating systems iOS and Android \cite{Majid2015}. The difference lies in the addition of a control close loop that allows real-time reading of data which can then update parameters to generate new insights quickly. Based on the explanation in Table \ref{table:vrar}. 

\section{Research Problems}
Deskripsikan dengan jelas permasalahan yang ingin diteliti pada tesis. Permasalahan berisi penjelasan dari Problem yang termuat pada judul penelitian. Deskripsi masalah sebaiknya dituliskan dengan gaya bahasa deskriptif. Pada contoh judul diatas, Problem-nya adalah tentang deteksi penyakit kanker, sehingga penulis disini dapat menceritakan tentang sulitnya pendeteksian penyakit kanker dengan mendeskripsikan faktor-faktor yang membuat sulit dalam pendeteksiannya. Uraian permasalahan yang baik manakala penulis berhasil menyakinkan kepada pembaca tentang seberapa tingkat urgensi dari permasalahan tersebut sehingga membuat pembaca yakin bahwa permasalahan tersebut membutuhkan solusi dari penelitian pada tesis penulis.
\section{Contribution}
The contribution of this research to the medical field, especially medical and nursing students
\begin{enumerate}
    \item Using augmented intelligence technology, facilitate students' understanding of the structure of human anatomy.
    \item Presenting interactive and interesting learning concepts to reduce boredom during practicum.
    \item Provide self-training for students to study and test their abilities through quiz mode and learning mode.
\end{enumerate}
The contribution of this research for researchers, especially in the field of immersive technology
\begin{enumerate}
    \item The use of AR technology can be applied in the medical field while still taking into account the dependencies of human health.
    \item As a guide for the application of Augmented Intelligence technology that can show human movements through embedded artificial intelligence.
    \item Provides insight into building a good user experience by measuring system performance through the PIECES framework.
\end{enumerate}
The contribution of this research for users outside the end-user
\begin{enumerate}
    \item As a means of entertainment in the field of Augmented Reality technology.
    \item As a means of exploration for users to learn independently of the structure of human anatomy.
     \item As a means of information AR technology can be implemented in various fields, not only as a game but also as a learning tool.
\end{enumerate}

\section{Aims}
Deskripsikan dengan jelas tujuan penelitian tesis yang diangkat. Tujuan tesis harus jelas, singkat, dan mengandung klaim orisinalitas. Tuliskan secara argumentatif apa saja fitur-fitur yang ditawarkan pada penelitian sebagai sesuatu solusi yang baru untuk mengklaim orisinalitas pada penelitian tesis. Untuk memberikan gambaran yang jelas apa yang dilakukan dan solusi unik apa yang akan ditawarkan pada penelitian, tujuan sebaiknya diawali dengan kalimat pembuka seperti ini: “Penelitian tesis ini mengajukan suatu pendekatan / algoritma / pemodelan / teknik / metode yang baru untuk mengatasi ........................ (hubungan dengan Problem) dengan memakai / menggunakan / mempresentasikan / melibatkan ..................... (sebutkan solusi memakai apa)”. Kalimat-kalimat selanjutnya kemudian memperjelas solusi dengan fitur-fitur unik seperti apa yang ditawarkan pada penelitian sebagai suatu solusi untuk menjawab permasalahan.
Pada contoh judul diatas, orisinalitasnya adalah Sistem Pakar Berbasis Fuzzy, sehingga penulis disini dapat menjelaskan tentang pemodelan sistem pakar dan fitur-fitur fuzzy yang seperti apa untuk deteksi penyakit kanker. Kalimat pertama pada tujuan dapat diawali dengan contoh berikut, “Penelitian tesis ini mengajukan suatu pendekatan baru untuk pendeteksian penyakit kanker dengan mempresentasikan model klasifikasi menggunakan Sistem Pakar yang berbasis Fuzzy”. Kemudian pada kalimat-kalimat berikutnya terangkan secara argumentatif tentang fitur-fitur unik pemodelan Sistem Pakar dengan Fuzzy sehingga dapat digunakan untuk mendeteksi penyakit kanker.

The aims of this research are:{\vspace{-2ex}}
\begin{enumerate}
       \item Create a clinical practice module using Augmented Intelligence technology installed on mobile devices.{\vspace{-2ex}}
       \item Implements healthcare scenarios into clinical practice modules for multiple operating system options. {\vspace{-2ex}}
       \item Implemented the facial anatomy structure selection feature on mobile devices.{\vspace{-2ex}}
       \item Analyzes changes in facial movements that can interact with visual and non-visual sensors to transmit data on the mobile device display.{\vspace{-2ex}}
\end{enumerate}

\section{Advantage}
Uraikan kontribusi tesis pada pengembangan ilmu pengetahuan teknologi dan seni, pemecahan masalah pembangunan, atau pengembangan kelembagaan. Kontribusi menggambarkan manfaat dari penelitian terhadap pihak tertentu saat penelitian sudah selesai. Kontribusi sebaiknya bersifat spesifik, tidak terlalu luas dan tidak terkesan mengada-ada. Jelaskan siapa yang mendapatkan manfaat dari penelitian penulis dan dalam bentuk apa manfaatnya.

\section{Writing System}

Jelaskan tentang sistematika pembahasan dalam buku tesis yang meliputi:
\\
CHAPTER 1 INTRODUCTION

Jelaskan tentang apa saja yang dibahas pada Bab 1. Penjelasan memuat bagian-bagian penting pada Pendahuluan.
\\
CHAPTER 2: LITERATURE STUDY

Jelaskan tentang apa saja yang dibahas pada Bab 2. Penjelasan memuat bagian-bagian penting pada Kajian Pustaka.
\\
CHAPTER 3: SYSTEM DESIGN AND DEVELOPMENT

Jelaskan tentang apa saja yang dibahas pada Bab 3. Penjelasan memuat bagian-bagian penting pada Desain Sistem.
\\
CHAPTER 4: SYSTEM TESTING AND ANALYSIS

Jelaskan tentang apa saja yang dibahas pada Bab 4. Penjelasan memuat bagian-bagian penting pada Eksperimen dan Analisis.
\\
CHAPTER 5: CONCLUSION

Jelaskan tentang apa saja yang dibahas pada Bab 5. Penjelasan memuat bagian-bagian penting pada Penutup.

\nomenclature[z-AIVE]{AIVE}{Artificial Intelligence in Virtual Education}
\nomenclature[z-SLAM]{SLAM}{Simultaneous Localization and Mapping}
\nomenclature[z-IMU]{IMU}{Inertia Measurement Unit}
\nomenclature[z-AR]{AR}{Augmented Reality}
\nomenclature[z-VR]{VR}{Virtual Reality}
\nomenclature[z-AI]{AI}{Artifical Inteligence}
\nomenclature[s-AuI]{AuI}{Augmented Intelligence}
\nomenclature[z-UI]{UI}{User Interface}
