\begin{abstrak}
    Keterampilan masyarakat semakin baik seiring dengan kemajuan teknologi, yang sejalan dengan kenyataan bahwa pendidikan semakin hari semakin baik. Salah satu caranya adalah melalui augmented intelligence, di mana manusia dan teknologi bekerja sama untuk membuat orang menjadi lebih pintar. Dalam penelitian ini, augmented intelligence digunakan untuk membantu orang belajar tentang anatomi manusia. Hal ini dilakukan dengan menggunakan teknologi augmented reality untuk melacak gerakan, yang dapat membuat objek 3D mengikuti gerakan. Platform untuk mempelajari anatomi disebut AIVE (Kecerdasan Buatan dalam Pendidikan Virtual). $\phi*\psi=g+opo 0.123{00303}$

Penelitian ini dibagi menjadi 3 mode utama, yaitu mode studi, mode ujian, dan mode pelacakan gerakan manusia. Setiap mode dapat dioperasikan melalui tombol yang 
\end{abstrak}